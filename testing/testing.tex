\documentclass{article}
\usepackage{amsmath}
\usepackage{graphicx}

\begin{document}

\section*{Figuring out what happens with the solver when there is
  attenuation}

I like to use simple test cases because they clarify to me what is
\textit{not} the problem. So I will be considering the following
situation: a $1\times 1\times 1$ meter box, and I will be evaluating
the solution at the center of that cube. The left face of the cube is
marked as a port, as is the right. So the equations we will be
solving when we are driving the left port are:
\begin{align*}
  -\omega^2 p(x,y,z) - c^2 \Delta p(x,y,z) &= 0, \\
  p(x=0,y,z) &= 1, && \text{(left boundary)}\\
  p(x=1,y,z) &= 0, && \text{(right boundary)} \\
    \frac{\partial p}{\partial n} &= 0,
     && \text{(all other boundaries)}
\end{align*}
The wave speed is computed as
\begin{align*}
  c = \sqrt{\frac{K}{\rho}},
\end{align*}
where $K$ is the bulk modulus and $\rho$ is the density, both of which
are read from the input file as frequency-dependent quantities, and
which can both be complex-valued.


\subsection*{Testing the pressure}

The set-up is chosen so that the solution is one-dimensional. As a
consequence, the solution only depends on $x$ and we can easily derive
it as
\begin{align*}
  p(x,y,z) = p(x) &= \frac{e^{jkx} - e^{-jk(2L-x)}}{1 - e^{-2jkL}},
\end{align*}
with $L=1$ and $k=\frac{\omega}{c}$. It is not difficult to verify
that indeed the boundary conditions are satisfied:
\begin{align*}
  p(0) &= \frac{e^{jk0} - e^{-jk(2L-0)}}{1 - e^{-2jkL}}
  =
  \frac{1 - e^{-2jkL}}{1 - e^{-2jkL}} = 1,
  \\
  p(L) &= \frac{e^{jkL} - e^{-jk(2L-L)}}{1 - e^{-2jkL}}
  = \frac{e^{jkL} - e^{-jkL}}{1 - e^{-2jkL}} = 0.
\end{align*}
It is also not difficult to check that the equation itself is
satisfied:
\begin{align*}
  -\omega^2 p(x) - c^2 \Delta p(x)
  &= -\omega^2 p(x) - c^2 p''(x)\\
  &= -\omega^2 p(x) - c^2 (jk)^2 p''(x)  \\
  &= (k^2c^2-\omega^2) p(x) \\
  &= \left(\frac{\omega^2}{c^2}c^2-\omega^2\right) p(x)\\
  &= 0.
\end{align*}

With this solution, evaluated at the center of the box of length
$L=1$, we obtain
\begin{align*}
  p_\text{center} &= p(0.5)
  \\
  &= \frac{e^{\frac{jk}{2}} - e^{-\frac{3jk}{2}}}{1 - e^{-2jk}}
  \\
  &= \frac{e^{\frac{jk}{2}} - e^{-\frac{3jk}{2}}}{e^{jk} - e^{-jk}}
  \\
  &= \frac{e^{\frac{j\omega}{2c}} - e^{-\frac{3j\omega}{2c}}}{1 - e^{-\frac{2j\omega}{c}}}
  \\
  &= \frac{e^{\frac{j\omega}{2}\sqrt{\frac{\rho}{K}}} - e^{-\frac{3j\omega}{2}\sqrt{\frac{\rho}{K}}}}{1 - e^{-2j\omega\sqrt{\frac{\rho}{K}}}}
\end{align*}
We will consider this expression for two special cases below, without
and with attenuation.


\subsubsection*{No attenuation}

Let us consider the evaluation of the pressure at the center of the
box for the special case without attenuation. For simplicity, we
choose $\rho=K=1$ and consequently $c=1$ and $k=\omega$. Then, based
on the formulas in the \texttt{readme.md} file at the top level of the
github repository, we have
\begin{align*}
  p(x)
  &=
  \frac{e^{jkx} - e^{-jk(2L-x)}}{1 - e^{-2jkL}}.
\end{align*}
This formula, after a good bit of massaging, can be restated as
follows for real-valued $k$:
\begin{align*}
  p(x)
  &=
  \frac{\sin(k(L-x))}{\sin(kL)}.
  \\
  &=
  \frac{\sin(\omega(1-x))}{\sin(\omega)}.
\end{align*}
This expression, unsurprisingly, has singularities for
$\omega=\pi,2\pi,3\pi,...$, where the cavity is in resonance. This
corresponds to frequencies $f=0.5, 1, 1.5, \ldots$ Hertz. As a
consequence, we better consider frequences below the first resonance,
i.e., $\omega<\pi$, i.e., $f<0.5$ Hertz.

That said, for the specific case of the center of the box (at $x=0.5$), we get
\begin{align*}
  p_\text{center}
  &= 
  \frac{\sin(\omega/2)}{\sin(\omega)}.
\end{align*}
In the low-frequency case, $\omega\ll 1$, we have the asymptotic
behavior $\sin(t)\approx t$ and so
\begin{align*}
  p_\text{center}
  &\approx
  \frac{\omega/2}{\omega}
  = \frac 12.
\end{align*}
This makes sense: In the low-frequency limit, the length of the cavity
is much less than the wavelength and so the pressure simply decreases
linearly from one at the left end to zero at the right end -- implying
a pressure of 0.5 at the center of the object.

We can plot what the program produces for a number of frequencies in
the range between 0.01 and 1 Hz:
%
% dii_frequency_response.csv, columns 1 (frequency), 14 (real part of
% pressure), 15 (imaginary part of pressure)
%

\begin{center}
\includegraphics[width=0.7\textwidth]{no-attenuation/pressure-at-center.png}
\end{center}

As can be seen, the computations (shown as the blue and green crosses) are an
excellent match for the expected theoretical behavior. Furthermore,
the singularity (resonance) is at $f=0.5$ Hz, again as expected.


\subsubsection*{With attenuation}

Let us now consider the case with attenuation. Specifically, we will
choose the same box of length $L=1$, but use $\rho=1-j, K=1$ and
consequently $c=\sqrt{\frac{K}{\rho}}=\sqrt{\frac{1}{1-j}}=\frac{1}{2^{1/4}}e^{j\pi/8}$ 
and $k=\omega/c=2^{1/4}e^{-j\pi/8} \omega$. Then, again based
on the formulas in the \texttt{readme.md} file at the top level of the
github repository, we have
\begin{align*}
  p(x)
  &=
  \frac{e^{jk(L-x)} - e^{-jk(L-x)}}{e^{jkL} - e^{-jkL}}.
\end{align*}
This time, because $k$ is not real valued, we cannot replace $e^{ja}$
by $\cos(a)+j\sin(a)$, and instead need to use the expression as
is. We can, however, simplify it to
\begin{align*}
  p(x)
  &=
  \frac{\sinh(jk(L-x))}{\sinh(jkL)}.
\end{align*}
Consequently,
\begin{align*}
  p_\text{center}
  &=
  \frac{\sinh(jk/2))}{\sinh(jk)}.
\end{align*}
We can again make the same observation that in the low-frequency
limit, $\sinh(t)=\frac 12(e^t-e^{-t})\approx \frac 12[(1+t)-(1-t)]=t$,
and so
\begin{align*}
  p_\text{center}
  &=
  \frac{\sinh(jk/2))}{\sinh(jk)}
  \approx
  \frac{jk/2}{jk}
  = \frac 12.
\end{align*}
The reasoning for this is as before: With or without damping, in the
low-frequency regime, the pressure is linear and so equal to one half
at the center of the domain.

We can again output the results of our computations:
\begin{center}
\includegraphics[width=0.7\textwidth]{with-attenuation/pressure-at-center.png}
\end{center}
As before, the results provide an excellent match to the theoretical expectations.

\end{document}
